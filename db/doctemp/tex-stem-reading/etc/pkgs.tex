% !TeX root = ../main.tex
% set package here
\PassOptionsToPackage{usenames,dvipsnames}{xcolor}
\PassOptionsToPackage{colorlinks=True,%
%linkcolor=,%used to disable color for TOC, but later solved by adding a group around tableofcontents command
filecolor=,%
citecolor=VioletRed,urlcolor=Blue,%
pdfborder={0 0 0}}{hyperref}

% Layout
\usepackage[margin=1cm,bmargin=2cm]{geometry} % a wide layout
\usepackage[font={small},skip=1pt]{caption}
\usepackage{parskip}
%%% for adding PDF bookmark
\usepackage{bookmark}

% STEM
\usepackage{amsmath,amssymb}
%\usepackage{wasysym}
%\usepackage{physics}

% Text
%\usepackage[normalem]{ulem}

% Figures
\usepackage{float}
\usepackage{wrapfig}
%\usepackage{overpic} % text over pictures
\usepackage{graphicx}
% path to search figures, each in a brace and endswith /
\graphicspath{{figures/}}
% subfigure and captions
%\usepackage{subcaption}
%\captionsetup{subrefformat=parens}

% Color
\usepackage{xcolor}

% Tabular
\usepackage{grffile}
\usepackage{booktabs}
%\usepackage{multirow}
%\usepackage{longtable}
%\usepackage{rotating}

%% Appendices package.
%% This is abandoned, as the appendices environment
%% will be wrongly put under the preceeding part.
%% Instead, manually set then appendix in appendix/main.tex
%%\usepackage[toc,page]{appendix}

% 解决相对路径input的问题
\usepackage{import}

% 编译器依赖的字体库导入
\usepackage{ifxetex}
\ifxetex%
\usepackage{xeCJK}%
\newcommand{\chinese}[2][gkai]{{#2}}%
\else%
\usepackage[T1]{fontenc}%
% pdflatex 下的简易中文环境. 字体支持少, 对此有要求的话还是用 xelatex
\usepackage{CJKutf8}%
\newcommand{\chinese}[2][gkai]{\begin{CJK*}{UTF8}{#1}#2\end{CJK*}}%
\fi

% === Coding related ===
%\usepackage[newfloat,cache=true]{minted}
%\definecolor{codebg}{rgb}{0.97,0.97,0.97}
%\setminted{mathescape,autogobble,breaklines,bgcolor=codebg,%
%style=xcode,fontsize=\footnotesize}
\makeatletter
\ltx@ifpackageloaded{minted}%
{% Do something if minted loaded
%Rename the listing environment
\ifxetex%
\renewcommand{\listingscaption}{清单}%
\renewcommand{\listoflistingscaption}{清单列表}%
\else\fi%
}%
{%Do something different if not
}%
\makeatother

% === Bibliogrphay by biblatex ===
%\usepackage[maxnames=3,style=phys,date=year,url=false,isbn=false,doi=false]{biblatex}
%articletitle=false
% link doi, issn or isbn to the pages (journal) or title (book)
\makeatletter
\ltx@ifpackageloaded{biblatex}%
{% Do something if biblatex loaded
    \newbibmacro{string+doiurlisbn}[1]{%
        \iffieldundef{url}{%
            \iffieldundef{doi}{%
                \iffieldundef{isbn}{%
                    \iffieldundef{issn}{%
                        #1%
                    }{%
                        \href{http://books.google.com/books?vid=ISSN\thefield{issn}}{#1}%
                    }%
                }{%
                    \href{http://books.google.com/books?vid=ISBN\thefield{isbn}}{#1}%
                }%
            }{%
                \href{https://dx.doi.org/\thefield{doi}}{#1}%
            }%
        }{%
            \href{\thefield{url}}{#1}%
        }%
    }
    % 只把 title 和 subtitle 改为 sentence case
    % https://tex.stackexchange.com/questions/22980/sentence-case-for-titles-in-biblatex
    \DeclareFieldFormat{sentencecase}{\MakeSentenceCase{#1}}
    \renewbibmacro*{title}{%
      \ifthenelse{\iffieldundef{title}\AND\iffieldundef{subtitle}}%
        {}%
        {\ifthenelse{\ifentrytype{article}\OR\ifentrytype{inbook}%
          \OR\ifentrytype{incollection}\OR\ifentrytype{inproceedings}%
          \OR\ifentrytype{inreference}\OR\ifentrytype{online}%
          \OR\ifentrytype{thesis}}%
          {\printtext[title]{%
            \printfield[sentencecase]{title}%
            \setunit{\subtitlepunct}%
            \printfield[sentencecase]{subtitle}}}%
          {\printtext[title]{%
            \printfield[titlecase]{title}%
            \setunit{\subtitlepunct}%
            \printfield[titlecase]{subtitle}}}%
         \newunit}%
      \printfield{titleaddon}}
    \DeclareFieldFormat[article,incollection]{pages}{\usebibmacro{string+doiurlisbn}{#1}}
    \DeclareFieldFormat[book]{title}{\usebibmacro{string+doiurlisbn}{\mkbibemph{#1}}}
}
{%
    % Do something different if not
}%
\makeatother

% === make indices ===
%\usepackage{imakeidx}
%\makeindex[columns=2, title=Index, intoc]
%% use \index{},\printindex
%% use intoc to make index appear in TOC
%% add \printindex at last to show the index

% === make glossaries ===
%\usepackage[acronym,toc]{glossaries}
%\makeglossaries
% change style, style=list,altlist,listgroup,listhypergroup
%\glossarystyle{style}
% term example:
%\newglossaryentry{latex}%
%{
%        name=latex,
%        description={A mark up language specially suited for
%scientific documents}
%}
% acronym example:
%\newacronym{gcd}{GCD}{Greatest Common Divisor}
%
% use \gls{}, \Gls{}, \glspl{}, \Glspl{} to print the term in lower/capital and single/plural
% add \printglossary[title=Glossary,toctile=Glossary] at last to show the glossary
% use \acrlong{}, \acrshort{}, \acrfull{} to print the acronym in "Aha Bba", "AB" and "Aha Bba (AB)" forms.
% add \printglossary[type=\acronymtype] at last to show the acronyms

\usepackage{hyperref}
%\usepackage{cleveref}

